\section{Discussion}

The implementation of the economic component of our hydro-economic model uses the stochastic and recursive data assimilation framework based on the ensemble Kalman filter proposed by Maneta and Howitt \citet{Maneta2014}, however the new implementation adopts the form of the optimality conditions for calibration proposed by \citet{Merel2011b} and \citet{Garnache2017}. The implementation of the positive mathematical programming methods proposed by these authors has two major advantages over the standard implementation used by \citet{Maneta2014}. One advantage is that it eliminates the need to solve an initial linear constrained optimization problem to identify the unknown Lagrange multipliers associated with land and water constraints \citep{Howitt1995}. Another important advantage is that it does not require a quadratic or exponential specification of the land cost function used in the standard implementation to provide the response function with the correct curvature \citep{Howitt1995, Howitt2012}.  

 Our analysis demonstrates that information on land use, crop evapotranspiration and crop production from existing and newer satellite-based remote sensing products contain sufficient information to calibrate the hydro-economic model presented here. It also shows that the recursive data assimilation methodology is effective for filtering out observation noise and identifying the correct model parameters within a relatively small number of assimilation cycles. Furthermore, The recursive updating nature of the filtering algorithm is ideal for model applications in non-stationary systems because it adapts the model calibration to the realities of the regional agriculture. Variations in the model parameters over time may be indicative of changes in the economic behavior of farmers triggered by external factors that may not be directly or easily detectable from satellite-based remote sensing information (e.g soil fertility) or due to shifting farmer perception or management practices. For instance, the recursive calibration for Beaverhead county (Figure \ref{fig:calibration2008-2015}), shows that parameter $\delta$, which represents the production returns to scale, is slowly increasing over time for irrigated alfalfa while remaining relatively stable for other crops. Values less than unity for this parameter indicate that crop production will increase exponentially less than a given increase in land and water allocated to this crop. Decreasing returns to scale reflect agricultural realities like the likelihood that crops expand to land of lessening quality, or the diminishing returns of additional water applications when crops are irrigated near their optimal level. Upward trends of this parameter signal technical or management improvements over our study period that increase productivity returns from land and water inputs. On the other hand, Parameter $mu$, which is a factor that represents the efficiency of the production system, seemed to decline for irrigated and non-irrigated spring wheat. Declines in this factor indicate loss of county production in these crops that cannot be explained by a reduction in the amount of land and water allocated to these crops. For instance, if may signal loss of soil fertility or declines in other inputs not explicitly included in the model. Note that the sensitivity of the model parameters to new observations is dependent on the value of $a$ in the parameter blending Eq. \eqref{eq:param_evolution}. This $a$ factor was prescribed at a value of 0.5 for this analysis, which provides a high level of smoothing and results in model parameter ensembles that represent long term conditions. Higher values of $a$ increase the calibration sensitivity to new observations, however at the expense of less stable parameter traces.
 
 An important characteristic of economic models of agricultural production calibrated using the PMP methodology is that there is no need to know with precision the production costs because the calibration algorithm approximates unknown or unspecified production costs by adjusting the $\lambda_{land,i}$ and $\lambda_{water,i}$. In our implementation of the optimality conditions, negative $\lambda_{i}$ values for land or water indicate that there are unobserved benefits associated with these inputs, which is to say that the observed production costs, $c_{land_i}$ and $c_{water,i}$, are overestimated. Conversely, a positive $\lambda_{i,l}$ value means that the observed production costs are under-estimated. In general, and in the case of Beaverhead county (Figure \ref{fig:calibration2008-2015}), irrigated crops typically have a negative value for $\lambda_{water,i}$, while positive values are more common in non-irrigated crops. 
 
 The capacity to estimate the value of $\lambda_{water,i}$ is an important characteristic of the calibration method, however this parameter is often the  most unstable during the data assimilation process and probably one of the largest sources of uncertainty in model predictions, as diagnosed by biases in component (2) of the innovation (Figure \ref{fig:innovation20082015}). Errors in the identification of this parameter may be the largest source of bias in the predictions of water and land allocation for years 2017 and 2018. 
 
 We found during our data assimilation experiments (not presented in this study) that the parameter ensembles are very sensitive to the uncertainty in some specific observations, especially observations of yield elasticity ($\overline{\pi}_W$) and supply elasticity ($\overline{\eta}$). If the noise to signal ratio in the observations was too high, the ensemble of model parameters converged to a biased solution even if the observation errors were unbiased, and this is one contributor to the prediction biases. \citet{Kanellopoulos2010} did similar forecast experiments using two variants of the positive mathematical programming method and reported model prediction sensitivities to the supply elasticity parameters and biases in model forecasts.  
 
 Our calibration methodology permits an unusual analysis of the model parameters and the quality of the model outputs. Commonly, hydro-economic models of agricultural production that are calibrated using any standard variation of the positive mathematical programming methodology are verified by reproducing the same baseline observations used for calibration. Model predictions under conditions different from those of the baseline are rarely verified with actual observations before the model is used in the simulation of design scenarios (\citep{Graveline2016}). This is because the empirical nature of the positive mathematical programming method limits the application of models calibrated using this methodology to conditions that are not too different from those of the calibration. Our results show that the model has forecasting skill even under the unusual flash drought conditions of 2017 and can correctly reproduce the spatial patterns of land and water allocation at county scales, albeit with some biases. Further research is necessary to understand the range of conditions under which the model still provides valid forecasts. 

 A major strength of our model is its ability to track the hydrologic impact of producer behavior. The hydrologic impacts of agricultural activity are maximal in early and mid August, right before crops mature and water diversions start to decline. This is also the period of lowest natural flows, therefore agricultural water diversions can exacerbate ecological stress in streams during years of low summer flows. However, the simulations also show that the temporal impacts of diversions are circumscribed to the irrigation season and streamflows recover quickly after diversions cease due to contributions from the substantial groundwater available in many of Montana's watersheds. The simulations also show that diversions also have an extensive spatial impact and their impact propagates downstream from counties where irrigation is most prevalent, such as these at the headwaters of the Missouri, Gallatin, and Yellowstone rivers. This pattern is correctly captured by the model. However, downstream impacts do not propagate unabated, and the recovery of streams is clearly visible in some reaches downstream of diversion nodes (Figure \ref{fig:map_water_change}. The recovery is caused by groundwater inflows into streams, which are known to be a key resilience mechanism for riverine ecosystems in the region. Groundwater is not yet extensively used as a source of water for irrigation in the region, and this may be the reason why the impacts of agricultural diversions are limited in space and in time. Conjunctive management of surface and groundwater may therefore be desirable to maintain the strength and resilience of Montana's rivers. 
 
 The results from the hydrologic component presented earlier are meant to illustrate the capabilities of the model, and should not be considered to be accurate. This model demonstration assumed 70\% efficiency in the water conveyance system and the irrigation technology for all counties. This efficiency parameter varies between counties and years and can have an important impact on the actual timing and volumes of diverted streamflows. Model refinements to improve the representation of these efficiencies are of major interest for state water managers and are underway. Another limitation of the current hydrologic component is that it does not yet include the effects of water impoundments. Artificial storage in dams and reservoirs and their release rules can have a large impact on the regional streamflow dynamics and provide additional resiliency by buffering and redistributing the spring freshet over a longer period. This model improvement is also a current priority.     

% Maybe talk some more about sensitivity and policy, like bottelneck regions and downstream disprotection if % rules do not limit upstream water use 
 