\section{Conclusions}

We describe the implementation of a hydro-economic model composed of a stochastic economic model of agricultural production and a rainfall-runoff and streamflow routing model that explicitly represents the spatial configuration of the regional water distribution network. The spatially-explicit nature of the hydrologic model permits to track the hydrologic impacts of producer activity. The economic component is designed to be continuously calibrated using remote sensing observations of land use, crop evapotranspiration and crop production. The calibration method is based on an implementationof the Positive Mathematical Programming methodology within a data assimilation framework that permits the recursive update of the model parameters when new remote sensing information becomes available.   The new formulation of the calibration methodology eliminates some of the limitations that have hindered the use of hydro-economic models to inform agricultural water management over large regions and temporal extents. Specifically, the model was designed to eliminate the need for expensive field surveys, reduce the problem of parameter overfitting to the conditions of the year and subset of farms used for calibration, and  reduce the false of precision that is associated with deterministic models.

The model is demonstrated for the state of Montana. We showed that satellite-based remote sensing retrievals of crop mix, land allocation, water allocation and crop yield, along with other ancillary information freely available online, contained sufficient information to correctly identify the parameters of the economic module. An interesting aspect of the recursive nature of the data assimilation methodology is that it permits to analyze the dynamics of the parameter ensembles over time, which may reveal trends in the biophysical and other factors that drive decision-making and are hard to observe directly, such as declines in land fertility, existence of hidden production costs, etc. The model was calibrated with eight years of observations (2008-2015) and effectively reproduced the observed levels of resource allocation of the 2008 baseline year used to spin-up the parameters. It also correctly predictive the spatial patterns of land and water allocation for years 2017 and 2018. Finally, we showed how the model can trace the effect of producer decision-making on the regional hydrologic system. This innovation of the model could be an important tool to understand how producer behavior affects the availability of water in the future, and how agricultural water use at county scales propagates through the hydrologic system to affect downstream users. The analysis of the time and propagation of streamflow drawdowns induced by agricultural water use may help identify regions that are at higher risk of water shortage during droughts.  