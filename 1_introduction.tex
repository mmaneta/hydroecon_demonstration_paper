\section{Introduction}

Many productive agricultural regions in the world are characterized by highly variable inter-annual precipitation, groundwater supplies, and stream flows. This variability is already increasing, and expected to continue an upward trend with climate change \citep{Groisman1994, Easterling2000, McCabe2005a, Mote2006b, Long2013}. Correspondingly, more frequent and intense droughts and more severe storm and runoff events will present new challenges for water managers \citep{Harou2006, Gorelick2015}. As opportunities to develop new water supplies decline, managers will need to improve the efficiency of the existing sources to satisfy growing demands \citep{USArmyCorpsofEngineers2012}. 

Agriculture has a long history of adapting to variability in local conditions \citep{McCarl2015, Rose2015}. Evidence to date suggests that farmers have met these challenges by changing their water allocation, crop mix, and land use \citep{Schneider, Bryant2000, Menzel2006}. However, little is known about how adaptation alters natural hydrologic systems and affects water users downstream, and how policy instruments may encourage or impede adaptation \citep{White2011}.

Regional resource managers rely on modeling tools to inform decision making, including hydro-economic models that simulate the balance between the regional water supply system and the anticipated demands from agricultural producers under a range of scenarios. Hydro-economic models are integrated tools that incorporate the realities of water management systems, including spatial impacts and dynamic demands driven by economic and policy drivers \citep{Harou2009b}. These types of models have been a subject of research since the late 1990s \citep{Pulido-Velazquez, Ward1996, Cai2003, Ward2006, Cai2008, Brouwer2008, Medellin-Azuara2011} and are becoming one of the most promising tools for integrated water management in the future. However, many of these operational water management tools are ultimately water accounting models and do not incorporate internal feedback mechanisms that alter the balance between the water supply and demand in the hydro-economic system. Another limitation of current operational water management tools is that they typically neglect the spatially explicit and dynamic nature of human actions, often assuming that the behavior of one farmer does not affect the choices of other farmers downstream. However, upstream decision-making is likely to influence the availability of water for downstream uses and the ability of downstream farmers to adapt to climate change \citep{Maneta2009e, Maneta2009c}.

Adaptive behavior and spatial-dynamic processes are rarely simulated because they are difficult to represent in models \citep{Aerts2018, Wens2019}. An efficient way to achieve this is to incorporate human behavior into hydro-economic models using a constrained optimization approach with farmer response functions calibrated to reflect observed decision making. This optimization approach is followed by models calibrated using the Positive Mathematical Programming (PMP) method \citep[][]{Howitt1995}. Models calibrated using PMP have been widely used to understand and optimize agricultural water allocation and for policy analysis \citep{Maneta2009c, Medellin-Azuara2008, Torres2011a, Ghosh2014, Kahil2016, Heckelei2013, Graveline2014, Connell-Buck2011, Medellin-Azuara2011, USBoR2011, DWR2009, Cobourn2011}, and can represent farmer behavior at a fraction of the complexity and computational requirements of other popular alternative approaches, such as statistical, econometric or agent-based models \citep{Wurster2019, Ng2011, Weersink2002}. An additional advantage of this approach is that the calibrated hydro-economic models are more amenable to coupling with physically-based models that represent the distributed regional hydrologic system. This coupling is key to tracing the effects of farmer adaptation on natural systems over space and time. 

PMP is a well-established method of calibrating hydro-economic models, but its predictive capability hinges on the quality and quantity of the data that it uses to reflect observed farmer behavior. The popularity of programming methods in operational hydro-economoic models has to some extent been limited by the availability of high-quality data for calibration, which is often derived from survey data on producer behavior. Due to the relatively high cost of administering surveys, data collection efforts necessary to calibrate and refine hydro-economic models models are often focused on specific watersheds, limiting the transferability and utility of these models. Among other problems, if the surveyed farms or the year of the survey are not representative of the group or the long term conditions, the calibration may have a bias toward farm conditions representing a particular survey period. An opportunity to overcome this limitation is to use satellite-based remote sensing observations of agricultural activity spanning multiple years of record. The increased availability of high spatial resolution remote sensing time series data allows for classification of crop types and determination of land allocation trends \citep{USDANASS2015}, retrieval of vegetation productivity including crops \citep{He2018, Mu2009} and estimation of vegetation water use \citep{He2019, Zhang2010, Allen2007} at a finer spatial and temporal scale, and across a broader geographic scope, than has been possible to date using survey instruments.

Although remote sensing data are subject to greater noise than survey data, remote sensing retrievals of agricultural activity provide continuous annual observations over a long period of time and over large geographic extents. Recent advances in data assimilation methods allows the use of recursive filters to compensate information quality with quantity and estimate model parameters using noisy but frequent observations of agricultural land and water allocations \citep{Maneta2014}. In this paper, we present and demonstrate a hydro-economic modeling framework that can be calibrated and applied over large regions by using recent advances in remote sensing and data assimilation methods to enable automatic model updates and calibration refinements. These innovations overcome current limitations of hydro-economic models and allow us to develop new insights into how farmers behave under resource and policy constraints. We demonstrate the implementation of the hydro-economic model and define its accuracy for the hydrologic and agricultural systems in the State of Montana. These systems span a representative range of conditions in the western U.S., including extensive dryland agriculture that is particularly vulnerable to climate variability. 

