\documentclass[review]{elsarticle}

\usepackage{lineno,hyperref}
\usepackage{amsmath}
\usepackage{siunitx}
\usepackage{xcolor}
\modulolinenumbers[5]

\journal{Environmental Modeling and Software}

%%%%%%%%%%%%%%%%%%%%%%%
%% Elsevier bibliography styles
%%%%%%%%%%%%%%%%%%%%%%%
%% To change the style, put a % in front of the second line of the current style and
%% remove the % from the second line of the style you would like to use.
%%%%%%%%%%%%%%%%%%%%%%%

%% Numbered
%\bibliographystyle{model1-num-names}

%% Numbered without titles
%\bibliographystyle{model1a-num-names}

%% Harvard
\bibliographystyle{model2-names.bst}\biboptions{authoryear}

%% Vancouver numbered
%\usepackage{numcompress}\bibliographystyle{model3-num-names}

%% Vancouver name/year
%\usepackage{numcompress}\bibliographystyle{model4-names}\biboptions{authoryear}

%% APA style
%\bibliographystyle{model5-names}\biboptions{authoryear}

%% AMA style
%\usepackage{numcompress}\bibliographystyle{model6-num-names}

%% `Elsevier LaTeX' style
%\bibliographystyle{elsarticle-num}
%%%%%%%%%%%%%%%%%%%%%%%

\begin{document}

\begin{frontmatter}

\title{DROUGHT IMPACT ASSESSMENT IN MONTANA USING A SATELLITE-DRIVEN HYDROECONOMIC MODEL}
%\tnotetext[mytitlenote]{Fully documented templates are available in the elsarticle package on \href{http://www.ctan.org/tex-archive/macros/latex/contrib/elsarticle}{CTAN}.}

%% Group authors per affiliation:
\author[mycorrespondingauthor]{Maneta M. P.\corref{mycorrespondingauthor}}
\author{Kimball, J}
\author{He, M}
\author{Silverman, N}
\address{Geosciences Department, University of Montana, Missoula, MT}
\fntext[myfootnote]{marco.maneta@umontana.edu}

\author{Maxwell, B}

\address{Land Resources and Environmental Sciences, Montana State University, Bozeman, MT}


\author{Cobourn, K}
\author{Ji, X}
\address{Department of Forest Resources and Environmental Conservation, Virginia Tech, Blacksburg, VA}

%% or include affiliations in footnotes:
%\author[mymainaddress,mysecondaryaddress]{Elsevier Inc}
%\ead[url]{www.elsevier.com}

%\author[mysecondaryaddress]{Global Customer Service\corref{mycorrespondingauthor}}
\cortext[mycorrespondingauthor]{Corresponding author}
%\ead{support@elsevier.com}

%\address[mymainaddress]{1600 John F Kennedy Boulevard, Philadelphia}
%\address[mysecondaryaddress]{360 Park Avenue South, New York}

\begin{abstract}
This template helps you to create a properly formatted \LaTeX\ manuscript.
\end{abstract}

\begin{keyword}
hydro-economic models, drought, agricultural systems
\end{keyword}

\end{frontmatter}

\linenumbers

\section{Introduction}

As climate change leads to unprecedented changes in natural systems, it is essential to encourage adaptation to offset the negative effects of those changes (Intergovernmental Panel on Climate Change 2014). Agriculture has a long history of adapting to variability in local conditions, but ongoing and substantial changes in climate are creating new challenges for farmers across the U.S. and globally \citep{McCarl2015, Rose2015}. Evidence to date suggests that farmers have met these challenges by adapting with changes in crop mix and land use, for example (Schneider et al. 2000; Bryant et al. 2000; Menzel et al. 2006). However, little is known about how farmers adapt; to what extent adaptation mitigates economic losses from climate change; how adaptation in turn alters natural systems; and how policy instruments may encourage or impede adaptation (White et al. 2011). 
Understanding how farmers adapt to changing natural conditions is critical for developing efficient policies to support producer welfare, enhance food security, and protect the environment. The objective of this project is to develop a transformative decision support tool that will allow policymakers and natural resource managers to understand the incentives that drive farmers’ adaptation to changing natural conditions, as well as the environmental consequences of adaptive behavior.  To do so, we propose a state-of-the-art integrated hydro-economic model that leverages recent advances in remote sensing science and in data assimilation methods to enable automatic model updates and refinements as new information on farming activity becomes available. Our methodology does not aim at optimizing the way farmers should allocate land, water, and other resources under different resource constraints. More importantly for policymaking is its capability to anticipate how they will allocate these resources, and the impact of their decisions on agricultural water demands, agricultural productivity and farm profits. Our methodology is designed for analysis at the farm or county scales and reveals how farmers react and reallocate resources seasonally when confronted with new climate, policy rules, or market signals. 
One reason current operational water management tools do not incorporate farmers’ behavior is because it is difficult to represent in models (Young et al. 1986). Agent-based models offer a very promising approach to simulate individual and collective action in response to changes in exogenous conditions. Unfortunately their complexity and computational requirements of these models limit their potential for operational water management and circumscribe their use to academic research (e.g. Yang et al. 2009 and references therein). Another limitation of current operational water management tools is that they often neglect the spatially explicit and dynamic nature of human action, often assuming that the behavior of one farmer does not affect the choices of farmers downstream. However, upstream decision-making is likely to influence the availability of water for downstream uses and the ability of downstream farmers to adapt to climate change (Maneta et al. 2009a; Maneta et al. 2009c).

Hydro-economic models, such as the modeling approach we propose, overcome these limitations, and provide a promising basis for an operational water management tool. Hydro-economic models are integrated tools that incorporate the realities of water management systems, including spatial impacts and dynamic demands driven by economic and policy drivers (Harou et al. 2009). These types of models have been a subject of research since the late 1990s (Pulido-Velazquez et al.; Ward and Lynch 1996; Cai et al. 2003; Ward et al. 2006; Cai 2008; Brouwer and Hofkes 2008; Medellín-Azuara 2011) and are becoming one of the foremost tools for integrated water management in the future. Our group has extensive experience with these methods and our previous work shows that integrated hydro-economic modeling can simulate farmer behavior at a fraction of the complexity and computational requirements of agent-based models (Maneta et al. 2009a; Maneta et al. 2009b; Maneta et al. 2009c; Torres et al. 2011; Ghosh et al. 2014). In addition, hydro-economic models are more amenable to coupling with physical models that represent the distributed regional hydrologic system. This coupling is key to tracing the effects of farmer adaptation on natural systems over space and time.

When applied to water management in agricultural regions, farmer behavior can be represented in hydro-economic models using response functions calibrated using positive mathematical programming (Howitt 1995). However, the predictive ability of these models is only as good as the quality of the behavioral observations used in the calibration. The availability of high-quality data for calibration has been limited by the availability of survey data on producer behavior. Data collection has often focused on specific watersheds, limiting the transferability and scope of these models. With the increased availability of high spatial resolution remote sensing data, we have an exciting opportunity to extend this approach to capture producer adaptation at a finer scale and across a broader geographic scope than has been possible to date.   

In this project, we propose to leverage and combine our diverse team’s previous work in multiple disciplines (see section 1.2) to develop and calibrate a stakeholder-informed, innovative, and integrated hydro-economic model that incorporates three major advances:

1) We will exploit new satellite-based remote sensing methods and products to calibrate our hydro-economic model. Specifically, we propose to ingest remote sensing data into a positive mathematic programming (PMP; Howitt 1995) approach to capture previously underrepresented factors that influence farmer decision-making. This will allow us to operationalize a hydro-economic model that spans a wider geographic scope than previous studies, with the potential to move hydro-economic modeling from the watershed to the sub-continental scale. 
2) We will implement a modeling approach based on recursive Bayesian inference that will permit us to evaluate the quality of the predictions based on the quality of the data (Maneta and Howitt 2014). It will also allow us to evaluate explicitly how changes in risk and uncertainty influence producer decision-making. This is a critical innovation given that the IPCC (2014) recently recognized that risk plays an important, and understudied, role in driving producer adaptation to climate change. 
3) We will trace the effect of producer decision-making on regional hydrologic systems. This innovation allows us to understand how behavior affects the availability of water in the future, and also how the water-use decisions of individuals propagate through the hydrologic system to influence the behavior of downstream water users. 

These innovations will advance hydro-economic modeling, overcome current limitations, and allow us to develop new insight into how famers behave under resource and policy constraints at unprecedented spatial extents and spatio-temporal resolutions. The resulting model will contribute to the next generation of decision support tools used in water policy analysis. Our proposed decision support tool is poised for use as a continuous, operational policy support system because the response of farmers to changing conditions can be continually updated using recursive inference methods from frequently available remote sensing information. This tool will support improved policy analysis to decide how to use water resources, including where to invest in water development infrastructure, how to design adaptation pathways, how to estimate water value, and how to manage water banks and other water marketing tools.
The model will be applied, and its accuracy defined, for Montana agricultural systems. These systems span a representative range of conditions in the western U.S., including extensive dryland agriculture that is particularly vulnerable to climate fluctuations, the existence of shared water governance with native groups and compacts with neighboring states, constraints imposed by legacy water rights and prior appropriation laws. 
1.1. Preliminary work 
Most models that integrate water resources and agricultural economics are composed of an economic optimization component linked to some type of hydrologic model that provides physical constraints on the amount of water and land available for agricultural production (Harou et al. 2009). Classic linear optimization models of agriculture implemented to simulate agent behavior often produce unrealistic results because it is not possible to explicitly account for all of the variables affecting farmer decisions. To overcome this problem, many modern economic optimization models used in policy analysis are based on a methodology called positive mathematical programming (PMP, Howitt (Howitt 1995)). PMP reduces the amount of data and artificial restrictions needed to calibrate classic optimization models. It also avoids overspecialization in crop production (Howitt 1995) and ensures that the model calibrates to observed conditions. 
An important characteristic of models calibrated using PMP is that they relate agricultural production (yield and profit) to agricultural input variables (e.g. crop mix, acreage, water applied) using observed farmer response, not the physiology of the crop or other agronomic information. This is important because, when calibrated this way, the model captures the actual behavior of farmers and their actual reaction to external factors, such as droughts and risk, rather than the behavior that would be optimal from a purely agronomic point of view. The economic behavior of farmers is motivated by a desire to maximize profit but also driven by culture, personal experience, and tradition, among other factors, which are often developed to reduce risk. These important features that drive the economic behavior of farmers are captured using the PMP methodology. Models based on PMP have been used intensively in drought analysis and policy design in California (Connell-Buck et al. 2011; Medellín-Azuara et al. 2011). These models are at the heart of the Statewide Agricultural Production Model (SWAP, Howitt et al. (Howitt et al. 2012)), which is used in the agricultural economic component of the CALVIN model of the California water system (Draper et al. 2003). SWAP is also used for policy analysis by the California Department of Water Resources (Department of Water Resources 2009) and the U.S. Department of the Interior’s Bureau of Reclamation (US Bureau of Reclamation 2011). Our team members have also used it to calibrate spatially-explicit, coupled hydrologic-economic models (Maneta et al. 2009a; Maneta et al. 2009c; Cobourn and Crescenti 2011). 
PMP is a well-established method of calibrating hydro-economic models, but its predictive ability hinges on the quality and quantity of the data that it uses to reflect observed farmer behavior. To date, PMP-based hydro-economic models are calibrated using costly survey data. This project will capitalize on the wealth of operational remote sensing data products and algorithms that are becoming available at little to no cost to improve the PMP calibration. Our team contains extensive experience in the design and application of remote sensing algorithms for regional to global scale assessment and monitoring of vegetation. Relevant experience includes the development and use of satellite microwave sensor based parameter retrievals for ecosystem studies (Jones et al. 2010; Jones and Kimball 2010; Jones et al. 2012); development of enhanced retrieval of plant biomass and phenology by fusing synergistic satellite optical-infrared and microwave remote sensing information (Kimball et al. 2009; Mu et al. 2009); documenting and improving the accuracy of satellite based time series of vegetation productivity (Heinsch et al. 2006; Kimball et al. 2007; Zhang et al. 2007) and evapotranspiration (Zhang et al. 2010). 
To take advantage of the high temporal frequency of remote sensing information we will leverage a second key enhancement of the PMP method (Maneta and Howitt 2014). This methodological extension permits recursive assimilation of remote sensing information using a stochastic data assimilation framework that overcomes two major limitations of classic PMP: 1) it now permits more robust model calibration by blending new and past information during the calibration process, avoiding over-calibration for the conditions of a single year; and 2) it generates predictions of resource allocation in terms of probability distributions that reflect the quality of the calibration and the uncertainty in the observations of agricultural activity.
The stochastic data assimilation framework is based on the equations of the ensemble Kalman filter (Evensen 2003) , which permits recursive updates of the mean and covariance of the economic model parameters as new observations of agricultural activity become available (Error: Reference source not found).  As more information is assimilated, the algorithm will improve the identification of the model parameters, which over time will converge to a distribution that reflects the quality of the observations being assimilated. Uncertainty in the model parameters and in the hydrologic conditions are also reflected in the prediction of resource allocation in the form of probability distributions (Error: Reference source not founda). Once the model is calibrated, it can be used to predict and study how farmers are likely to reallocate resources under various conditions by running the model under different scenarios of crop prices and costs of agricultural inputs, or under different levels of water and land restrictions due to environmental or policy factors (Error: Reference source not foundb). The probability distribution of the model parameters also permits an interpretation that provides insight into how risk influences producer decision-making and adaptation. We propose to build on this model component to identify and understand how the variance of critical economic and hydrologic parameters affects producer decision-making about resource use and the corresponding

\section{Model Description}

\section{Economic model of agriculture and farmer behavior}

Farmer behavior is represented by a response functions calibrated using positive mathematical programming, PMP \citep{Howitt1995}). The backbone of the component is a maximization an economic model of agricultural production   

\section{Hydrologic component}

The hydrologic component provides water availability constraints to agricultural production. Precipitation is transformed into runoff using a gridded version of the HBV model \citep{Bergstrom1973, Bergstrom1995, Lindstrom1997}. When runoff reaches the channel it is routed through the stream network using the Muskingum-Cunge method, which was applied to flow routing by \cite{Cunge1969}. Both models are very well known and tested, and because of its parsimonious nature, reliability, robustness, and performance they have been widely applied in many regions of the world for hydrologic response analysis under climate change and drought \citep{Driessen2010, Menzel2002}, flood    forecasting \citep{}  and water management \citep{}. 

HBV is a precipitation-runoff model originally developed to assist in flood forecasting in Sweden. The hydrologic system is conceptualized as a cascade of four compartments: snowpack, soil, upper groundwater zone, and lower groundwater zone in each of the hydrologic response units (HRUs) in which the user may divide the region. Water outputs from the soil and groundwater compartments of each HRU are transformed into runoff using a convolution with a triangular unit hydrograph. The model was implemented as a partially gridded version of the original model. It requires daily precipitation inputs, maximum and minimum daily air temperature and produces the runoff response of the different subcatchments composing the study area. The model also tracks four internal states in each of the subcatchments: snow water equivalent, soil water storage, water storage in the upper groundwater compartment and water storage in the deep water compartment. The model has \num{12} tuning parameters. Details of the model structure and implementation are provided in \ref{app:hydrologic_model}. 

The runoff response of each subcatchment becomes lateral flows into the stream reach contained in the subcatchment. Lateral runoff and inflows from upstream subcatchments are routed using the Muskingum-Cunge model. This Muskingum model uses a two-parameter constitutive equation to relate storage ($S$) in a reach to its inflows ($Q_{in}$) and outflows ($Q_{in}$): $S = K\left[eQ_{in} + (1 - e)Q_{out}\right]$, where $K$ and $e$ are the two function parameters.  This permits to substitute out storage from the mass balance equation for the reach. The Muskingum-Cunge method uses this relationship to develop a finite difference approximation of the 1D diffusion equation. 
Full details on the Muskingum-Cunge algorithm and its implementation  are provided in \ref{app:hydrologic_model}. 

\subsection{Model coupling}

The model components are internally coupled and information between them is transferred at different points of the simulation cycle through common fluxes. At the beginning of the water year, the economic model is run in simulation mode and the total land and water that farmers choose to allocate to each crop for the growing season is determined for each economic region. This information is used to calculate the water that needs to be diverted from the hydrologic network at daily time steps. This is done by redistributing the total crop water requirements over the growing season according to its phenology as represented by the crop coefficients......  

\begin{align}
    q^t_j_{irr} = \frac{Wat}{I_{eff}}
\end{align}


The hydrologic component operates at daily time steps and at the spatial resolution of the climate grid and HRUs, while the economic model of agricultural production operates at annual time steps and at spatial resolutions defined by the user, typically administrative units such as counties.  

\section{Data assimilation}

\subsection{Remote sensing retrievals of agricultural activity}


Bayesian estimation

\section{Application to Montana}




\section{Discussion}

\section{Conclusions}

\section*{References}

\bibliography{library}

\appendix
\section{Hydrologic model}
\label{app:hydrologic_model}

The hydrologic system is simulated using a rainfall-runoff model coupled to a routing component that simulates streamflows in the regional stream network. We adapted the HBV model \citep{Bergstrom1995, Bergstrom1973} to simulate subcatchment-scale hydrologic processes (snowmelt, evapotranspiration, infiltration) and to transform precipitation into runoff and streamflow. Runoff that reaches the channel is routed through the stream network using the Muskingum-Cunge routing algorithm \cite{Chow1988}. In this appendix we provide here a description of the implementation of the algorithms.

\subsection{Rainfall Runoff component}

The HVB model \citep{Bergstrom1995, Bergstrom1973} is implemented as a mixture of gridded and vector-based operations to leverage the distributed nature of raster meteorological datasets while simultaneously taking advantage of the reduced computational burden of operating over polygons that aggregate runoff production over uniform hydrologic response units (HRUs). 

Snowpack accumulation and melt and soil processes are calculated over the uniform raster grid imposed by the meteorological inputs (precipitation, air temperature, and potential evapotranspiration). In the next two paragraphs subscript $i$ indicates that the variable or parameter is spatially distributed and is represented at grid point $i$. Superscript $t$ indicates that the variable is dynamic and its value is represented at time step $t$. Variables with no script or superscript indicate that the variable is spatially constant or time invariant. 

\paragraph{Precipitation and snowpack processes}      


Precipitation is partitioned between snowfall and rainfall using minimum and maximum daily air temperatures and a critical temperature threshold $Tc$ that determines the the snow-rain transition:

\begin{align}
Snow_i^t &= \left\{
	\begin{array}{ll}
	P_i^t &  Tmax_i^t < Tc_i \\   
	P_i^t * \frac{Tc_i - Tmin_i^t}{Tmax_i^t - Tmin_i^t} & Tmin_i^t < Tc_i < Tmax_i^t \\
	0 &  Tmin_i^t > Tc_i \\
	\end{array}
\right.\\
Rain_i^t &= P_i^t - Snow_i^t 	 
\end{align} 
\noindent where $P$ is precipitation (\si{\milli\metre\per\day}), $T_{max}$ and $T_{min}$ are maximum and minimum air temperature (\si{\degreeCelsius}), $Rain$ is liquid precipitation and $Snow$ is snowfall at pixel $i$ during time step $t$ (\si{\milli\metre\per\day}). Snowfall during day $t$ contributes to the snow water equivalent ($SWE$, (\si{\milli\metre})) of the snowpack:

\begin{equation}
SWE_i^t = SWE_i^{t-1} + Snow_i^t \Delta t
\end{equation}

The snowpack melt process is simulated using a degree day factor model occurs when average air temperature exceeds a air temperature threshold ($Tm$):

\begin{align}
Melt_i^t &= ddf_i * (Tav_i^t - Tm_i)  ]\text{ for } Tav_i^t > Tm_i \\
Rain_i^t &= P_i^t - Snow_i^t 	 
\end{align} 

\noindent where $Melt$ is the amount of water output from the snowpack (\si{\milli\metre\per\day}), $Tav$ is average air temperature over the time step (\si{\degreeCelsius}), and $ddf$ is the degree day factor (\si{\milli\metre\per\day\per\degreeCelsius}), an empirical parameter that represents the snowmelt rate per degree of air temperature above $Tm$. Any melt form the snowpack during time $t$ is subtracted from the snowpack storage ($SWE$) and added to the amount of water ponded in the surface: 

\begin{align}
Pond_i^t &= Pond_i^{t-1} + (Melt_i^t + Rain_i^t)\Delta t \\
SWE_i^t &= SWE_i^t - Melt_i^t \Delta t 	 
\end{align} 

\noindent where $Pond$ (\si{\milli\metre}) is liquid water available on the surface to infiltrate or produce runoff. 

\paragraph{Soil processes}  
Recharge into the soil system occurs when liquid water ponding the surface infiltrates into the soil. Ponded water that is not infiltrated increases the topsoil compartment that generates fast runoff. The fraction of ponded water that infiltrates into the soil is a exponential function of the relative water storage in the soil: 

\begin{align}
\Delta SM_i^t &= Pond_i^t * \left(1 - \frac{SM_i^t}{FC_i^t} \right)^\beta \\
\end{align} 

\noindent where $SM$ (\si{\milli\metre}) is the amount of water in the soil compartment, $FC$ (\si{\milli\metre}) is the maximum amount of water soil can hold before water starts percolating to the groundwater system, and $beta$ (dimensionless) is an empirical parameter. Simultaneously, actual evapotranspiration ($AET$, \si{\milli\metre\per\day}) reduces the amount of water storage in the soil and is also controlled by the degree of saturation of the soil (ration of $SM$ to $FC$). 

\begin{align}
AET_i^t &= PET_i^t * \left(\frac{SM_i^t}{FC_i * LP_i} \right)^l	 \\
\end{align} 

\noindent where $PET$ is potential evapotranspiration (\si{\milli\metre\per\day})) and $l$ is an empirical dimensionless parameter. Infiltration and actual evapotranspiration control the dynamics of water storage in the soil and amount of surface water that generates fast runoff:

\begin{align}
SM_i^t &= SM_i^{t} + \Delta SM_i^t - AET_i^t \Delta t\\
OVL_i^t &= Pond_i^t - \Delta SM_i^t
\end{align}

\noindent where $OVL$ (\si{\milli\metre}) is water that recharges the upper (near-surface) runoff-generating compartment.

\paragraph{Percolation and runoff generation}
Excess water in the topsoil and in two groundwater compartments generate outflow that represent fast and intermediate runoff and baseflow. These processes are implemented at the HRU level. For this, calculations about overland flow generation and soil moisture performed at the grid level are averaged over subwatersheds representing HRUs. Spatial arithmetic averaging soil water storage over all grid cells $i$ contained within a given HRU $j$ is represented using angle brackets $<.>$. The mass balance and percolation of water from the soil upper to the soil lower zone is implemented as:

\begin{align}
Rech_j^t &= <OVL_i^t>_j + <max(SM_i^t - FC_i, 0)>_j\\
SUZ_j^t &= SUZ_j^{t-1} + Rech_j^t + Pond_j^t - Q0_j^t\Delta t - Q1_j\Delta t - PERC_j\\
SLZ_j^t &= SLZ_j^{t-1} + PERC_j - Q2 \Delta t
\end{align}

\noindent $Rech$ (\si{\milli\meter}) is water storage in the near-surface compartment that generates fast runoff, $SUZ$ (\si{\milli\meter}) is the storage in the upper groundwater compartment, and $SLZ$ (\si{\milli\meter}) is water storage in the lower (deeper) groundwater compartment in HRU $j$ at time step $t$. $Q_0$, $Q_1$, and $Q_2$ (\si{\milli\meter\per\day}) are specific runoff rates from the soil surface, and the upper and lower soil zones:

\begin{align}
Q0_j^t &= max((SUZ_j - HL1_j) * \frac{1}{CK0_j}, 0.0)\\ 
Q1_j^t &= SUZ_j * \frac{1}{CK1_j}\\
Q2_j^t &= SLZ_j * \frac{1}{CK2_j}\\
Qall_j^t &= Q0_j^t + Q1_j^t + Q2_j^t
\end{align}

\noindent where $HL1$ (\si{\milli\meter}) is an empirical water storage threshold the triggers the generation of fast runoff, and $CK0$, $C10$, $CK2$ (\si{\day}) are empirical parameters representing the characteristic drainage time of each of the compartments. 
Total outflow from HRU $j$ on day $t$ is distributed over time to produce the catchment response by convoluting the output of HRU $j$ by triangular standard unit hydrograph with base $M_{base}$. 

\begin{align}
Q_j^t &= \sum_{i=1}^{M_{base}} Qall_j^{t-i+1} U(i) \\
U(i) &= \left\{
\begin{array}{ll}
\frac{4}{M_{base}^2}*i & 0 < i < M_{base}/2 \\
-\frac{4}{M_{base}^2}*i + \frac{4}{M_{base}} &  M_{base}/2 < i < M_{base} \\
\end{array}
\right.
\end{align}

\noindent where $U$ is a triangular hydrograph of area \num{1} and a base $MAXBAS$ (\si{\day})representing the hydrograph duration .  

\subsection{Routing component}

The response at the end of each $HRU$ is routed through the stream network using the Muskingum-Cunge routing model. In this model the storage in each stream reach $k$ is given by the following discharge-storage equation:

\begin{align}
S_k^t &= K\left[eQ_{in} + (1 - e)Q_{out} \right],
\end{align}

which has parameters $K$ (\si{\day}) and $e$ (dimensionless) controlling, respectively, the celerity and dispersion of the wave routed through the channel. 

Substituting this relationship in a finite-difference form of the continuity equation $\frac{S_j^{t+1} - S_j^{t}}{\Delta t} = Q_{in} - Q_{out}$ for a multi-reach system with lateral inflows injected upstream of reach draining $HRU$ $j$ at average constant rate through time step $t$ $q_{j}^{t+1}$ yields:

\begin{align}
&Q_j^{t+1}\left[K_j(1 - e_j) + 0.5\Delta t  \right] + Q_{j-1}^{t+1}\left[K_je_j - 0.5\Delta t  \right]  \\
&= Q_j^{t}\left[K_j(1 - e_j) - 0.5\Delta t  \right] + Q_{j-1}^{t}\left[K_je_j + 0.5\Delta t  \right]\\
&+ q_{j}^{t+1}\left[K_j(1 - e_j) + 0.5\Delta t  \right]
\end{align}

Each of the $HRUs$ contains one reach with an upstream and a downstream node. Streamflows for each of the $j=1,...,J$ reaches are integrated over time using a first-order explicit finite difference scheme. The system of $J$ equations can be assembled as a linear system of the form:    

\begin{align}\label{eq:linearsystem}
\mathbf{A}\mathbf{Q^{t+1}} = \mathbf{B} 
\end{align}

where $\mathbf{Q^{t+1}}$ is the vector of unknown streamflows at time $t+1$ for each of the $J$ reaches of the network that is solved each time step. Matrices $\mathbf{A}$ add $\mathbf{B}$ are functions of the model parameters and streamflows at timestep $t$:
\begin{align}
\mathbf{A}&\equiv (\mathbf{a} + \mathbf{\Phi} \mathbf{b})^T\\
\mathbf{B}&\equiv (\mathbf{d} + \mathbf{\Phi}\mathbf{c})^T\mathbf{Q}^t + \mathbf{I}(\mathbf{a\odot q}^{t+1})
\end{align}

where $\mathbf{\Phi}$ is a $JxJ$ sparse connectivity (0,1)-matrix where the elements indicate if two pairs of nodes are connected. Flow direction is from nodes in the rows to nodes in the columns. Rows representing the upstream node of $HRUs$ that drain an outlet node (exit the domain) are all zero. Finally,

%\begin{align}
%(\mathbf{a} + \mathbf{\Phi} \mathbf{b})\mathbf{Q^{t+1}} = (\mathbf{d} + %\mathbf{\Phi}\mathbf{c})\mathbf{Q^t} + \diag(\mathbf{a})*\mathbf{q^{t+1}}
%\end{align}

\begin{align}
\mathbf{a} &= \mathbf{I} (\mathbf{K}-\mathbf{K\odot e}) + dt * 0.5\\
\mathbf{b} &= \mathbf{I} (\mathbf{K\odot e}) - dt * 0.5\\
\mathbf{c} &= \mathbf{I} (\mathbf{K}-\mathbf{K\odot e}) - dt * 0.5)\\
\mathbf{d} &= \mathbf{I} (\mathbf{K\odot e}) + dt * 0.5
\end{align}

\noindent where $\mathbf{K}$ is the identity matrix of order $J$, $\mathbf{K}$ and $\mathbf{e}$ are column vectors holding parameters $K$ and $e$ for each of the $N$ reaches in the network. The $\odot$ operator denotes the Schur (elementwise) product between two vectors.
The solution of \eqref{eq:linearsystem} becomes unstable if $\Delta t > 2 * K_j * (1 - e_j)$. To ensure robust and stable solution an adaptive time stepping scheme was implemented. In this scheme, the default time step is reduced by an integer fraction until the the stability condition is satisfied in all reaches. 


%\appendix
\section{Economic model}
\textcolor{red}{An intro paragraph on PMP. To be added.}


\subsection{Calibration}
The fundamental aspect of PMP is that farmers allocate resources with the objective of maximizing net revenues: 
\begin{equation}\label{eq:econ_simulation}
    \begin{split}
    \max _ { x _ { i 1 } , x _ { i 2 } \geq 0 } & \quad \sum _ { i } \left\{ p _ { i } \mu _ { i } \left[ \beta _ { i 1 } x _ { i 1 } ^ { \rho _ { i } } + \beta _ { i 2 } x _ { i 2 } ^ { * \rho _ { i } } \right] ^ { \frac { \delta _ { i } } { \rho _ { i } } } - \left( c _ { i 1 } + \lambda _ { i 1 } \right) x _ { i 1 } - \left( c _ { i 2 } + \lambda _ { i 2 } \right) x _ { i 2 } \right\} \\
    \text{subject to} & \quad \sum _ { i } x _ { i 1 } \leq \overline { L } \left[ \overline { \lambda } _ { 1 } \right]
    \end{split}
\end{equation}
where $i = 1,...,I$ indexes crops; $l=1,2$ indexes inputs (input 1 is land, input 2 is water). Define $x _ { i 2 } ^ { * } = x _ { 20 } + x _ { i 2 }$, where $x_{20}$ is natural ET and $x_{i2}$ is applied ET (irrigation). 

The unknown parameters in Equation~\ref{eq:econ_simulation} are $\mu _ { i } , \beta _ { i l } , \rho _ { i } , \delta _ { i } , \lambda _ { i l } , \overline { \lambda } _ { 1 }$; and the known values are:

\begin{itemize*}
    \item $\overline { \eta } _ { i }$: exogenous supply elasticity (\% change in supply over \% change in crop price)
    \item $\overline { q } _ { i }$: observed crop production
    \item $\overline { x } _ { i 1 }$: observed land allocation
    \item $\overline { x } _ { i 2} ^{*}$: observed ET (total)
    \item $\overline { y } _ { i W}$: reference yield elasticity with respect to water (\% change in production over \% change in total ET)
    \item $\sigma_i$: elasticity of substitution
    \item $p_i$: crop price
    \item $c_{il}$: per-unit input costs of production
    \item $\overline { L }$: land constraint (total arable land)
\end{itemize*}

Given the above maximization problem, the problem can be calibrated according to the following calibration equations: 

\begin{equation}
    \begin{split}
        & \rho _ { i } = \frac { \sigma _ { i } - 1 } { \sigma _ { i } } \text { for } i = 1 , \ldots , I \\
        & \overline { \eta } _ { i } = \frac { \delta _ { i } } { 1 - \delta _ { i } } \left\{ 1 - \frac { \frac { b _ { i } } { \delta _ { i } \left( 1 - \delta _ { i } \right) } } { \sum _ { j } \left[ \frac { b _ { j } } { \delta _ { j } \left( 1 - \delta _ { j } \right) } + \frac { \sigma _ { j } b _ { j } \overline { y } _ { j W } } { \delta _ { j } \left( \delta _ { j } - \overline { y } _ { j W } \right) } \right] } \right\} \text { for } i = 1 , \ldots , I , \text {where } b _ { i } = \frac { \overline { x } _ { i 1 } ^ { 2 } } { p _ { i } \overline { q } _ { i } } \\
        & \overline { y } _ { i W } = \delta _ { i } \left( \frac { \beta _ { i 2 } \overline { x } _ { i 2 } ^ { * \rho _ { i } } } { \beta _ { i 1 } \overline { x } _ { i 1 } ^ { \rho _ { i } } + \beta _ { i 2 } \overline { x } _ { i 2 } ^ { * } \rho _ { i } } \right) \text { for } i = 1 , \ldots , I \\
        & \sum _ { l } \beta _ { i l } = 1 \text { for } i = 1 , \ldots , I \\
        & \overline { q } _ { i } = \mu _ { i } \left[ \beta _ { i 1 } \overline { x } _ { i 1 } ^ { \rho _ { i } } + \beta _ { i 2 } \overline { x } _ { i 2 } ^ { * \rho _ { i } } \right] ^ { \frac { \delta _ { i } } { \rho _ { i } } } \text { for } i = 1 , \ldots , I \\
        & \overline { \lambda } _ { 1 } = \frac { \sum _ { i } \left[ p _ { i } \overline { q } _ { i } \left( \delta _ { i } - \overline { y } _ { i W } \right) - c _ { i 1 } \overline { x } _ { i 1 } \right] \overline { x } _ { i 1 } } { \sum _ { i } \left( \overline { x } _ { i 1 } ^ { 2 } \right) } \\
        & p _ { i } \overline { q } _ { i } \left( \delta _ { i } - \overline { y } _ { i W } \right) = \left( c _ { i 1 } + \lambda _ { i 1 } + \overline { \lambda } _ { 1 } \right) \overline { x } _ { i 1 } \text { for } i = 1 , \ldots , I \\
        & p _ { i } \overline { q } _ { i } \overline { y } _ { i W } = \left( c _ { i 2 } + \lambda _ { i 2 } \right) \overline { x } _ { i 2 } ^ { * } \text { for } i = 1 , \ldots , I
    \end{split}
\end{equation}

\subsection{Simulation}
The simulation problem is given by the following maximization problem:
\begin{equation}\label{eq:econ_simulation}
\begin{split}
    \max _ { x _ { i 1 } , x _ { i 2 } \geq 0 } &  \sum _ { i } \left\{ p _ { i } \mu _ { i } \left[ \beta _ { i 1 } x _ { i 1 } ^ { \rho _ { i } } + \beta _ { i 2 } x _ { i 2 } ^ { * } \rho _ { i } \right] ^ { \frac { \delta _ { i } } { \rho _ { i } } } - \left( c _ { i 1 } + \lambda _ { i 1 } \right) x _ { i 1 } - \left( c _ { i 2 } + \lambda _ { i 2 } \right) x _ { i 2 } \right\} \\
    \text{subject to } & \sum _ { i _ { i 1 } } x _ { i 1 } \leq \overline { L } \left[ \lambda _ { 1 } \right]; \sum _ { i } x _ { i 2 } ^ { * } \leq \overline { W } \left[ \lambda _ { 2 } \right]; \text {and } x _ { i 2 } \geq 0 \left[ \xi _ { i } \right]
\end{split}
\end{equation}
where the unknown values are $x _ { i 1 } , x _ { i 2 } , q _ { i } , \lambda _ { 1 } , \lambda _ { 2 } , \xi _ { i }$, and the known values include:
\begin{itemize}
    \item $\mu _ { i } , \beta _ { i l } , \rho _ { i } , \delta _ { i } , \lambda _ { i l }$: parameters determined from calibration
    \item $p _ { i } , c _ { i l } , \overline { L }$: prices, costs, and arable land constraint (same as in calibration)
    \item $\overline { W }$ : water availability constraint (from hydrologic model)
\end{itemize}

And the simulation equations are given by:
\begin{equation}
\begin{split}
    & \frac { p _ { i } \delta _ { i } q _ { i } \beta _ { i 1 } x _ { i 1 } ^ { \rho _ { i } } } { \beta _ { i 1 } x _ { i 1 } \rho _ { i } + \beta _ { i 2 } x _ { i 2 } ^ { * \rho _ { i } } } = \left( c _ { i 1 } + \lambda _ { i 1 } + \lambda _ { 1 } \right) x _ { i 1 } \text { for } i = 1 , \ldots , I \\
    & \frac { p _ { i } \delta _ { i } q _ { i } \beta _ { i 2 } x _ { i 2 } ^ { * \rho _ { i } } } { \beta _ { i 1 } x _ { i 1 } ^ { \rho _ { i } } + \beta _ { i 2 } x _ { i 2 } ^ { * } \rho _ { i } } = \left( c _ { i 2 } + \lambda _ { i 2 } + \lambda _ { 2 } + \xi _ { i } \right) x _ { i 2 } ^ { * } \text { for } i = 1 , \ldots , I \\
    & q _ { i } = \mu _ { i } \left[ \beta _ { i 1 } x _ { i 1 } ^ { \rho _ { i } } + \beta _ { i 2 } x _ { i 2 } ^ { * \rho _ { i } } \right] ^ { \frac { \delta _ { i } } { \rho _ { i } } } \text { for } i = 1 , \ldots , I \\
    & \sum _ { i } x _ { i 1 } = \overline { L } \text { for } i = 1 , \ldots , I \\
    & \sum _ { i } x _ { i 2 } ^ { * } = \overline { W } \text { for } i = 1 , \ldots , I
\end{split}
\end{equation}
Note that this system of equations requires that the water constraint is binding. If the water constraint is non-binding, we need to build in complementary slackness conditions that allow for the case in which $\lambda _ { 2 } = 0$.

\vspace{1in}
%%%%%%%%
\large
\textcolor{red}{Marco: Equations have been converted from the word document to here. I'm not sure if the rest of the documents need conversion too. They look to me like more of a technical note than a methodological guide in the appendix. Let me know what you think.}

MARCO: Thanks. I think that this should be good for the time being. I will probably use the calibration equations in the main part of the text. I am focusing on the simulations for the paper now and I have started writing a proposal, so I will put this in the back burner for a few days. 






\end{document}


